% This must be in the first 5 lines to tell arXiv to use pdfLaTeX, which is strongly recommended.
\pdfoutput=1
% In particular, the hyperref package requires pdfLaTeX in order to break URLs across lines.

\documentclass[11pt]{article}

% Change "review" to "final" to generate the final (sometimes called camera-ready) version.
\usepackage{acl}

% Standard package includes
\usepackage{times}
\usepackage{latexsym}

% For proper rendering and hyphenation of words containing Latin characters (including in bib files)
\usepackage[T1]{fontenc}

% This assumes your files are encoded as UTF8
\usepackage[utf8]{inputenc}

\usepackage{amsmath}

% This is not strictly necessary, and may be commented out,
% but it will improve the layout of the manuscript,
% and will typically save some space.
\usepackage{microtype}

% This is also not strictly necessary, and may be commented out.
% However, it will improve the aesthetics of text in
% the typewriter font.
\usepackage{inconsolata}

%Including images in your LaTeX document requires adding
%additional package(s)
\usepackage{graphicx}

\title{Etude du meilleur classifieur pour le DEFT de 2009}

\author{
  \text{MANSERI Kéhina\textsuperscript{(1)(2)}}
  \text{SIRVEN-VIENOT Alix\textsuperscript{(1)(2)}}
  \text{VAN-DEN-ZANDE Débora\textsuperscript{(1)(3)}}
\\
\\
  \textsuperscript{(1)}Université Paris Nanterre
  \textsuperscript{(2)}Parcours Recherche et Développement
  \textsuperscript{(3)}Parcours Pro
\\
\\
    \small {
    manserikehina@gmail.com, alix.vienot@gmail.com, vdzdebora@gmail.com
    }
\\
}

\begin{document}
\maketitle
\begin{abstract}
Coucou bouh abstract ! On va expliquer pleins de trucs mais pour l'instant on sait pas ce qu'on fait donc no lol.

Si on veut citer le texte on peut faire comme ça \cite{forest2009variation} et je trouve ça très stylé.

\end{abstract}

\section{Introduction}
On va faire des citations en bas de page comme ça\footnote{\url{http://acl-org.github.io/ACLPUB/formatting.html}} youpi !

Dans cet article, nous présentons nos résultats pour la recherche 

\section{Engines}

\section{Preamble}

\section{Document Body}

\subsection{Tables and figures}

\begin{table}
  \centering
  \begin{tabular}{lc}
    \hline
    \textbf{Nom} & \textbf{Taille de zeub} \\
    \hline
    Iris& -5 cm         \\
    Loic& 3 cm          \\
    Théo& 15 cm         \\
    Débora& 42 cm         \\\hline
  \end{tabular}
  
  \caption{La taille de zeub aux personnes associées.}
  \label{tab:accents}
\end{table}

As much as possible, fonts in figures should conform
to the document fonts. See Figure~\ref{fig:experiments} for an example of a figure and its caption.

\begin{figure}[t]
  \includegraphics[width=\columnwidth]{example-image-golden}
  \caption{imge zeubi}
  \label{fig:experiments}
\end{figure}

\begin{figure*}[t]
  \includegraphics[width=0.48\linewidth]{example-image-a} \hfill
  \includegraphics[width=0.48\linewidth]{example-image-b}
  \caption {A minimal working example to demonstrate how to place
    two images side-by-side.}
\end{figure*}

\subsection{Hyperlinks}


\subsection{Citations}


\subsection{References}

\subsection{Appendices}

Use \verb|\appendix| before any appendix section to switch the section numbering over to letters. See Appendix~\ref{sec:appendix} for an example.

\section{Références}
\bibliography{custom}
\cite{forest2009variation}

\appendix

\section{Exemple d'annexe}
Dans l'annexe on peut mettre pleins de choses, ça compte pas dans le nombre de pages.


\end{document}
